%%=============================================================================
%% Samenvatting
%%=============================================================================

% TODO: De "abstract" of samenvatting is een kernachtige (~ 1 blz. voor een
% thesis) synthese van het document.
%
% Een goede abstract biedt een kernachtig antwoord op volgende vragen:
%
% 1. Waarover gaat de bachelorproef?
% 2. Waarom heb je er over geschreven?
% 3. Hoe heb je het onderzoek uitgevoerd?
% 4. Wat waren de resultaten? Wat blijkt uit je onderzoek?
% 5. Wat betekenen je resultaten? Wat is de relevantie voor het werkveld?
%
% Daarom bestaat een abstract uit volgende componenten:
%
% - inleiding + kaderen thema
% - probleemstelling
% - (centrale) onderzoeksvraag
% - onderzoeksdoelstelling
% - methodologie
% - resultaten (beperk tot de belangrijkste, relevant voor de onderzoeksvraag)
% - conclusies, aanbevelingen, beperkingen
%
% LET OP! Een samenvatting is GEEN voorwoord!

%%---------- Nederlandse samenvatting -----------------------------------------
%
% TODO: Als je je bachelorproef in het Engels schrijft, moet je eerst een
% Nederlandse samenvatting invoegen. Haal daarvoor onderstaande code uit
% commentaar.
% Wie zijn bachelorproef in het Nederlands schrijft, kan dit negeren, de inhoud
% wordt niet in het document ingevoegd.

\IfLanguageName{english}{%
\selectlanguage{dutch}
\chapter*{Samenvatting}
\lipsum[1-4]
\selectlanguage{english}
}{}

%%---------- Samenvatting -----------------------------------------------------
% De samenvatting in de hoofdtaal van het document

\chapter*{\IfLanguageName{dutch}{Samenvatting}{Abstract}}
Dit onderzoek richt zich op het ontwerpen en ontwikkelen van een privacy behoudend Collaborative Filtering (CF) aanbevelingssysteem, waarbij Federated Learning (FL) en Differential Privacy (DP) worden geïntegreerd om gepersonaliseerde aanbevelingen te leveren zonder de privacy van gebruikers in gevaar te brengen. In tegenstelling tot traditionele CF-systemen, die afhankelijk zijn van gecentraliseerde datasets, worden in dit systeem gebruikersdata lokaal op hun eigen apparaten bewaard. Via FL worden alleen geaggregeerde modelupdates, verrijkt met zorgvuldig berekende ruis door middel van DP, naar een centraal globaal model gestuurd. Deze ruis wordt toegevoegd aan de parameters die tijdens het federale leerproces worden gedeeld, waardoor de privacy van individuele gebruikers wordt gewaarborgd zonder de nauwkeurigheid van het aanbevelingssysteem significant aan te tasten.

Het doel van dit onderzoek is het ontwikkelen van een proof of concept (POC) dat aantoont hoe FL en DP effectief kunnen worden gecombineerd om een privacy-behoudend CF-systeem te creëren. Hierbij ligt de focus op het vinden van een optimale balans tussen de kwaliteit van de aanbevelingen en het niveau van gegevensbescherming. Verwacht wordt dat deze aanpak de privacyrisico's van traditionele aanbevelingssystemen aanzienlijk vermindert, terwijl de relevantie en precisie van de aanbevelingen behouden blijven.

Dit onderzoek draagt bij aan een innovatieve oplossing die toepasbaar is in diverse sectoren waar gepersonaliseerde aanbevelingen worden ingezet. Bovendien heeft deze aanpak het potentieel om het vertrouwen van gebruikers te vergroten en de bedrijfswaarde te verhogen door een transparante en veilige verwerking van gevoelige data.
